\documentclass{article}
\usepackage[utf8]{inputenc}
\usepackage{geometry}
\usepackage{multicol}

\addtolength{\oddsidemargin}{-.875in}
\addtolength{\evensidemargin}{-.875in}
\addtolength{\textwidth}{1.75in}
\addtolength{\topmargin}{-.875in}
\addtolength{\textheight}{1.75in}

\title{408D Course Guide}
\author{Adrian Melendez Relli}
\date{Spring 2019}

\begin{document}

\maketitle




\section{Derivatives}
\begin{multicols}{3}

$$ \frac{d}{dx} \mathrm{sin(x) = cos(x)}$$
$$ \frac{d}{dx} \mathrm{cos(x) = -sin(x)}$$
$$ \frac{d}{dx} \mathrm{tan(x) = sec^2(x)}$$
$$ \frac{d}{dx} \mathrm{sec(x) = sec(x)\:tan(x)}$$
$$ \frac{d}{dx} \mathrm{cot(x) = -csc^2(x)}$$
$$ \frac{d}{dx} \mathrm{csc(x) = -csc(x)cot(x)}$$


\end{multicols}




\section{Integration}
\begin{multicols}{3}

$$ \int \mathrm{sin(x)} \:\mathrm{d}x = \mathrm{-cos(x)} + C $$ 
$$ \int \mathrm{cos(x)} \:\mathrm{d}x = \mathrm{sin(x)} + C $$
$$ \int \mathrm{tan(x)} \:\mathrm{d}x = \mathrm{-ln|cos(x)| + C} $$
$$ \int \mathrm{cot(x))} \:\mathrm{d}x = \mathrm{ln|sin(x)| + C} $$
$$ \int \mathrm{sec(x))} \:\mathrm{d}x = \mathrm{ln|sec(x) + tan(x)| + C} $$
$$ \int \mathrm{csc(x))} \:\mathrm{d}x = \mathrm{-ln|csc(x) + cot(x)| + C} $$

\end{multicols}

$$\mathrm{Integrate\:by\:Parts } \hspace{10mm}\int u \:\mathrm{d}v - \int v \:\mathrm{d}u$$






\section{Product of sines and cosines}

$$ \int sin^n(x) cos^m(x) $$

\begin{multicols}{2}

$$ \mathrm{Case \: 1: \: \: m = 1} \hspace{3mm}\frac{sin^{n+1}(x)}{n+1} + C $$ 

$$\mathrm{Case \: 2: \: \: \: m = odd} $$ 

\centerline{Use cos$^2$(x) = 1 - sin$^2$(x) to convert} 

\centerline{all but one power of cosine to sine.}


\medskip{}

$$ \mathrm{Case \: 3: \:  n = 1} \hspace{3mm}- \frac{cos^{m+1}(x)}{m + 1} + C $$ 

$$\mathrm{Case \: 4: \: \: \: n = odd} $$ 


\centerline{Use cos$^2$(x) = 1 - sin$^2$(x) to convert}

\centerline{all but one power of sine to cosine.}

\medskip{}

\bigskip{}


\centerline{Case 5: Both are even, Use Double Angle Formulas}

$$sin^n(x) cos^m(x) = (\frac{1}{2}(1 - cos(2x)))^{\frac{n}{2}} (\frac{1}{2}(1 + cos(2x)))^{\frac{m}{2}}$$

\end{multicols}






\section{Product of secants and tangents}

$$ \int sec^n(x) tan^m(x) $$

\begin{multicols}{2}

$$ \mathrm{Case \: 1: \: \: n = 2} \: \: \: \: u = tan(x), \int u^m du$$
 
\centerline{Case 2: n = even \hspace{3mm} Convert all but two}
\centerline{secants to tangents}

$$ \mathrm{Case \: 3: \: \: m = 1} \: \: \: \: u = sec(x), \int u^{n-1} du$$

\centerline{Case 4: m = odd \hspace{3mm} Convert all }
\centerline{but one power of tangent into secants}

\bigskip{}

\end{multicols}

\centerline{Be sure to factor out any coefficient inside trig function, e.g. $ cos(3x) \: \mathrm{becomes} \: \frac{1}{3}\int~ (\mathrm{u-sub})$}






\section{Trigonometric Formulas}

\begin{multicols}{2}


\centerline{Double Angle Formulas:}

\medskip{}

$$sin(2x) = 2 sin(x) cos(x)$$

$$cos(2x) = cos^2(x) - sin^2(x)$$

$$cos(2x) = 1 - 2sin^2(x)$$

$$cos(2x) = 2cos^2(x) - 1$$

\centerline{Products of Trig Functions with Different Angles}

$$sin(A)sin(B) = \frac{cos(A-B) - cos(A + B)}{2}$$

$$cos(A)cos(B) = \frac{cos(A-B) + cos(A + B)}{2}$$

$$sin(A)cos(B) = \frac{sin(A + B) + sin(A - B)}{2}$$

\end{multicols}




\section{Trig Substitution}



\end{document}
